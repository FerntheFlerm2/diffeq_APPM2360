\section{Conclusion} \label{sec:conc}

In this project, we analyzed a model of the Mariana Trench, with data from the NOAA. We first did a basic investigation wherein we graphed a representation of the depth data (matrix $\mtrA$), found the deepest point in the trench, and found the average depth of the trench. Then, we described the process of Singular Value Decomposition, and how Incomplete Singular Value Decomposition could be used to reduce the memory needed to represent a data set. Thereafter, we described a method to find the largest eigenvalue in a matrix, and used that algorithm to find the largest eigenvalue in $\mtrA^T\mtrA$. Next, we modified that method to find that largest $i$ eigenvalues of a matrix (where $i$ is an arbitrary constant) and used the resulting algorithm to find the largest $50$ eigenvalues of $\mtrA^T\mtrA$. Afterwards, we explained how to use these eigenvalues to construct the ISVD of a matrix, and used this method to generate numerous ISVD representations of the Mariana Trench, using different numbers of eigenvalues. Then, we quantified how much fewer values are needed to represent the depth of the Mariana Trench using ISVD. Penultimately, we demonstrated that using $50$ eigenvalues, the ISVD representation of the Mariana Trench is very close to the real data, meaning further analysis can be conducted on the ISVD representation without significantly affecting the quality of results. Finally, we showed how the fidelity of the ISVD representation increases as the number of eigenvalues used increases, in turn causing the total number of values used to increase. 

